\documentclass[16pt]{article}
\usepackage[utf8]{inputenc}
\usepackage{hyperref}
\usepackage[margin=0.5in]{geometry}
\title{Learning Agenda Week 11 July 15}
\author{ }
\date{ }

\begin{document}

\maketitle

\section * {Introduction To Probability and Statistics}

The Concepts of Probably - Learning Outcomes from the Course Outline:

6.5 State the addition law of probabilities

6.6 State what is meant by independent events and determine if events are independent

6.7 State the multiplication law of probabilities

6.8 Solve simple probability problems using complementary events and/or the addition/multiplication laws

Collect, organize and interpret data using tables, graphs, and simple statistical measures.

5.1 Set up data in a grouped and ungrouped frequency distribution

5.2 Draw a histogram and relative cumulative frequency curve

5.3 Define and contrast sample and population

5.4 Calculate the mean, mode, and median of a set of data

5.5 Define outliers and discuss their impact on data summary

5.6 Calculate the range of a set of data

5.7 Calculate the sample and population standard deviation of a set of data



%\url{https://www.udemy.com/probability-and-statistics-for-business-and-data-science/learn/lecture/10741736#overview}

\section * {Measures of Central Tendancy}

Mean

Median

Mode 

Measurements of Dispersion

Range

Sample Variance
Population Variance

Standard Deviation

\section * {Quartiles}
\section * {Data sets}
\subsection   {Univariant}
\subsection   {Bivariant}

\section{Introduction to Wolfram Alpha }

\url{https://www.wolframalpha.com}
\url{http://Wolframone.com}

\begin{enumerate}
    \item Create a Wolfram ID
    \item: Importing Data into Mathematica \url{https://youtu.be/MmS3JNk7JE4}
    \begin{verbatim}
    fileImp=Import["c:\\d\\DATA.csv"]
    \end{verbatim}
    \item   Getting help with using Wolfram Commands
\end{enumerate}
  
\section{Assignment: Computational Mathematics: Worth 3\% of your final grade}  
\subsection{The handin work is to be presented in Latex, submitted to GitHub}

In this assignment, you will take a Dataset of Real Estate Data and calculate the basic measures of Data Centrality including Standard Deviation

\end{document}
